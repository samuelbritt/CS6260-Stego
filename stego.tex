\documentclass[11pt]{article}

\usepackage{amsmath}
\usepackage{amssymb}
\usepackage{amsfonts}
\usepackage{amsthm}
\usepackage{mathtools}
\usepackage{xcolor}
\usepackage{graphicx}
\usepackage[alsoload=binary]{siunitx}
\usepackage{booktabs}
\usepackage{multirow}
\usepackage{fancyvrb}
\usepackage[section]{placeins}
\usepackage{flafter}
\usepackage{microtype}
\usepackage{url}
\usepackage{hyperref}

\usepackage{mathpazo}

% a bit more compact
\renewcommand\l{\mathopen{}\left}
\renewcommand\r{\right}

\newcommand\abs[1]{\l\vert #1 \r\vert}

% no section numbers
\setcounter{secnumdepth}{-2}

% leave notes to yourself
\newcommand\todo[1]{\textcolor{red}{\textsc{todo}: #1}}

%%% BLACKBOARD SYMBOLS

\newcommand{\C}{\ensuremath{\mathbb{C}}}
\newcommand{\D}{\ensuremath{\mathbb{D}}}
\newcommand{\F}{\ensuremath{\mathbb{F}}}
\newcommand{\G}{\ensuremath{\mathbb{G}}}
\newcommand{\J}{\ensuremath{\mathbb{J}}}
\newcommand{\N}{\ensuremath{\mathbb{N}}}
\newcommand{\Q}{\ensuremath{\mathbb{Q}}}
\newcommand{\R}{\ensuremath{\mathbb{R}}}
\newcommand{\T}{\ensuremath{\mathbb{T}}}
\newcommand{\Z}{\ensuremath{\mathbb{Z}}}
\newcommand{\QR}{\ensuremath{\mathbb{QR}}}

\newcommand{\Zt}{\ensuremath{\Z_t}}
\newcommand{\Zp}{\ensuremath{\Z_p}}
\newcommand{\Zq}{\ensuremath{\Z_q}}
\newcommand{\ZN}{\ensuremath{\Z_N}}
\newcommand{\Zps}{\ensuremath{\Z_p^*}}
\newcommand{\ZNs}{\ensuremath{\Z_N^*}}
\newcommand{\JN}{\ensuremath{\J_N}}
\newcommand{\QRN}{\ensuremath{\QR_{N}}}
\newcommand{\QRp}{\ensuremath{\QR_{p}}}

%%% GENERAL COMPUTING

\newcommand{\bit}{\ensuremath{\set{0,1}}}
\newcommand{\pmone}{\ensuremath{\set{-1,1}}}

% asymptotics
\DeclareMathOperator{\poly}{poly}
\DeclareMathOperator{\polylog}{polylog}
\DeclareMathOperator{\negl}{negl}
\newcommand{\Otil}{\ensuremath{\tilde{O}}}

\renewcommand{\epsilon}{\varepsilon}

% sets in calligraphic type
\newcommand{\calD}{\ensuremath{\mathcal{D}}}
\newcommand{\calF}{\ensuremath{\mathcal{F}}}
\newcommand{\calH}{\ensuremath{\mathcal{H}}}
\newcommand{\calX}{\ensuremath{\mathcal{X}}}
\newcommand{\calY}{\ensuremath{\mathcal{Y}}}

% types of indistinguishability
\newcommand{\compind}{\ensuremath{\stackrel{c}{\approx}}}
\newcommand{\statind}{\ensuremath{\stackrel{s}{\approx}}}
\newcommand{\perfind}{\ensuremath{\equiv}}

% font for general-purpose algorithms
\newcommand{\algo}[1]{\ensuremath{\mathsf{#1}}}
% font for general-purpose computational problems
\newcommand{\problem}[1]{\ensuremath{\mathsf{#1}}}

%%% "LEFT-RIGHT" PAIRS OF SYMBOLS

% "left-right" pairs of symbols

%% NOTE: this requires \usepackage{mathtools} in the document preamble

% inner product
\DeclarePairedDelimiter\inner{\langle}{\rangle}
% absolute value
% \DeclarePairedDelimiter\abs{\lvert}{\rvert}
% a set
\DeclarePairedDelimiter\set{\{}{\}}
% parens
\DeclarePairedDelimiter\parens{(}{)}
% tuple, alias for parens
\DeclarePairedDelimiter\tuple{(}{)}
% square brackets
\DeclarePairedDelimiter\bracks{[}{]}
% rounding off
\DeclarePairedDelimiter\round{\lfloor}{\rceil}
% floor function
\DeclarePairedDelimiter\floor{\lfloor}{\rfloor}
% ceiling function
\DeclarePairedDelimiter\ceil{\lceil}{\rceil}
% length of some vector, element
\DeclarePairedDelimiter\length{\lVert}{\rVert}
% "lifting" of a residue class
\DeclarePairedDelimiter\lift{\llbracket}{\rrbracket}

%%% CRYPTO-RELATED NOTATION

% KEYS AND RELATED

\newcommand{\key}[1]{\ensuremath{#1}}

\newcommand{\pk}{\key{pk}}
\newcommand{\vk}{\key{vk}}
\newcommand{\sk}{\key{sk}}
\newcommand{\mpk}{\key{mpk}}
\newcommand{\msk}{\key{msk}}
\newcommand{\fk}{\key{fk}}
% \newcommand{\id}{id}
\newcommand{\keyspace}{\ensuremath{\mathcal{K}}}
\newcommand{\msgspace}{\ensuremath{\mathcal{M}}}
\newcommand{\ctspace}{\ensuremath{\mathcal{C}}}
\newcommand{\tagspace}{\ensuremath{\mathcal{T}}}
\newcommand{\idspace}{\ensuremath{\mathcal{ID}}}

\newcommand{\concat}{\ensuremath{\|}}

% GAMES

% advantage
\newcommand{\advan}{\ensuremath{\mathbf{Adv}}}

% different attack models
\newcommand{\attack}[1]{\ensuremath{\text{#1}}}

\newcommand{\atk}{\attack{atk}} % dummy attack
\newcommand{\kr}{\attack{kr}}   % key recovery
\newcommand{\indcpa}{\attack{ind-cpa}}
\newcommand{\indcca}{\attack{ind-cca}}
\newcommand{\anocpa}{\attack{ano-cpa}} % anonymous
\newcommand{\anocca}{\attack{ano-cca}}
\newcommand{\euacma}{\attack{eu-acma}} % forgery: adaptive chosen-message
\newcommand{\euscma}{\attack{eu-scma}} % forgery: static chosen-message
\newcommand{\suacma}{\attack{su-acma}} % strongly unforgeable

% ADVERSARIES
\newcommand{\attacker}[1]{\ensuremath{\mathcal{#1}}}

\newcommand{\Adv}{\attacker{A}}
\newcommand{\AdvA}{\attacker{A}}
\newcommand{\AdvB}{\attacker{B}}
\newcommand{\Dist}{\attacker{D}}
\newcommand{\Sim}{\attacker{S}}
\newcommand{\Ora}{\attacker{O}}
\newcommand{\Inv}{\attacker{I}}
% \newcommand{\For}{\attacker{F}}

% CRYPTO SCHEMES

\newcommand{\scheme}[1]{\ensuremath{\text{#1}}}

% pseudorandom stuff
\newcommand{\prg}{\algo{PRG}}
\newcommand{\prf}{\algo{PRF}}
\newcommand{\prp}{\algo{PRP}}

% symmetric-key cryptosystem
\newcommand{\skc}{\scheme{SKC}}
\newcommand{\skcgen}{\algo{Gen}}
\newcommand{\skcenc}{\algo{Enc}}
\newcommand{\skcdec}{\algo{Dec}}

% public-key cryptosystem
\newcommand{\pkc}{\scheme{PKC}}
\newcommand{\pkcgen}{\algo{Gen}}
\newcommand{\pkcenc}{\algo{Enc}} % can also use \kemenc and \kemdec
\newcommand{\pkcdec}{\algo{Dec}}

% digital signatures
\newcommand{\sig}{\scheme{SIG}}
\newcommand{\siggen}{\algo{Gen}}
\newcommand{\sigsign}{\algo{Sign}}
\newcommand{\sigver}{\algo{Ver}}

% message authentication code
\newcommand{\mac}{\scheme{MAC}}
\newcommand{\macgen}{\algo{Gen}}
\newcommand{\mactag}{\algo{Tag}}
\newcommand{\macver}{\algo{Ver}}

% key-encapsulation mechanism
\newcommand{\kem}{\scheme{KEM}}
\newcommand{\kemgen}{\algo{Gen}}
\newcommand{\kemenc}{\algo{Encaps}}
\newcommand{\kemdec}{\algo{Decaps}}

% identity-based encryption
\newcommand{\ibe}{\scheme{IBE}}
\newcommand{\ibesetup}{\algo{Setup}}
\newcommand{\ibeext}{\algo{Ext}}
\newcommand{\ibeenc}{\algo{Enc}}
\newcommand{\ibedec}{\algo{Dec}}

% trapdoor functions
\newcommand{\tdf}{\scheme{TDF}}
\newcommand{\tdfgen}{\algo{Gen}}
\newcommand{\tdfeval}{\algo{Eval}}
\newcommand{\tdfinv}{\algo{Invert}}
\newcommand{\tdfver}{\algo{Ver}}

%%% PROTOCOLS

\newcommand{\out}{\text{out}}
\newcommand{\view}{\text{view}}

%%% COMMANDS FOR HOMEWORKS

\newcommand{\hwheader}{%
  \chead{\Large \textbf{Homework \hwnum}}

  \lhead{\small
    \textbf{\href{https://t-square.gatech.edu/portal/site/XLS0814164958201208.201208}{Applied
    Cryptography}\\Georgia Tech, Fall 2012}}

  \rhead{\small \textbf{Instructor:
      \href{http://www.cc.gatech.edu/~cpeikert/}{Chris
      Peikert}\\Student: \studentname}}

  \setlength{\headheight}{20pt}
  \setlength{\headsep}{16pt}
  
  \headrule
}


\newcommand\channel{\ensuremath{\mathcal C}}
\newcommand\sschac{\attack{ss-cha-$\channel$}}

% Steganography
\newcommand\stg{\scheme{S}}
\newcommand\stgenc{\algo{SE}}
\newcommand\stgdec{\algo{SD}}

\usepackage{tikz}

\title{Analysis of Stegosystems}
\author{Sam Britt, Tushar Humbe, Ben Perry, Sanchita Vijayvargiya}
\date{December 7, 2012}

\begin{document}
\maketitle

\section{Introduction}
Steganography effectively hides a message within plain sight, wherein
a message being communicated is hidden inside another ostensibly
innocent message. The message is imperceptible to the adversary that
acts as a regulator within the communication channel. Only the sender
and the receiver situated at either ends of the communication channel
are capable of deciphering existence of the hidden message.

The concept of steganography has been in use since the times of
ancient Greece. Communicating messages concealed underneath the wax of
writing tablets and the technique of dotting successive letters are
some of the ways in which the Greeks steganography. Pirate lore
includes tales of individuals tattooing secret information, such as
maps, on their head which would be covered by their hair. Throughout
World War 2 the grill method was used by the spies. This method
involved wooden templates which would be placed over seemingly
harmless text, revealing the secret message. During the same period,
the Germans devised microdot technology in which a picture could be
printed clearly even after shrinking it to the size of a dot.  At the
time of the British Rule in India, freedom fighters, in order to
communicate with each other used lemon extract to write hidden
messages on a paper containing general information. The hidden message
became visible only when the paper was heated by exposure to a candle
or the sun.

Gustavus Simmons, in 1984, initiated the scientific research on
steganography in ``The Prisoner's Problem and the Subliminal
Channel''[3] where he illustrated the concept of steganography with the
help of Prisoner's Problem. Two prisoners, Alice and Bob, attempt to
devise a plan to escape even though they are locked up in areas and
are prohibited from engaging in private communication with each other.
Both of the prisoners can correspond merely through a single
communication channel which is scrupulously monitored by the warden of
the prison. The warden, Wendy, is the adversary in this example and is
trying to intercept Alice and Bob's private communications. If Alice
and Bob try to exchange messages that are not completely open to
Wendy, or ones that seem suspicious to her, they will be put into a
high security prison forestalling their plan to escape. Messages sent
without any hidden message are known as covertexts. Messages sent on
the channel containing a hiddentext are known as stegotexts. At this
point, Alice and Bob make use of steganography, by sharing a secret
method or key to hide hiddentexts in some stegotext. Only the two
accomplices are able to decipher the hidden message based on their
secrets, while Wendy remains completely oblivious of that message.

In modern digital steganography, the message is hidden in a digital
file yielding a stegotext. This broad definition is useful because any
message can be embedded in any sort of digital file whether it is
text, photo, or other multimedia. Because this definition does not
define the method in which hiddentext is stored in a stegotext we can
define any method to do the job. However, not all steganographic
methods are steganographically secure. Imagine Alice sending a text
file to Bob and renaming the file extension. At first glance this
could seem like a reasonable message for Alice to send. Upon further
investigation Wendy will be able to identify the hidden message or the
fact that there is a hidden message by comparing this file to other
files of that extension.

In this paper we are going to go over what is needed in a
steganographic model in the first section. After understanding the
model that must be used in stegosystems we will look at the notion of
steganographic security. Next we will look at steganalysis, the act of
analyzing suspected stegotexts to verify the suspicion or obtain the
message. Before concluding, we will look at stegosystems and discuss
issues that these may have in terms of our notions of security.

\section{Steganographic Model}
There are variations in steganographic models, but most papers will
agree that a good model needs to include three parties and a monitored
communication channel. The three active members of the model are
Alice, Bob, and Wendy. Prisoners Alice and Bob are trying to hide a
message, hiddentext, from Wendy the warden. The warden monitors the
communication on the channel and inspects many messages on the
channel. Valid channel messages are known as covertext, these are
messages that should not arouse any suspicion from Wendy. Alice will
run an embedding algorithm on her hiddentext and some covertext to
generate her stegotext. Bob will run an extracting algorithm to obtain
the hiddentext from the stegotext. These are all of the needed pieces
in steganography. If Alice and Bob are able to pass a message which
includes hiddentext through this monitored channel without Wendy
becoming aware of the presence of hiddentext, then Alice and Bob have
successfully sent a stegotext. If Alice and Bob are not trying to hide
a message then they are just sending covertext. The below model
illustrates steganography in a similar manner to what is described
above.

\begin{figure}[htbp]
  \label{fig:model}
  \centering
  \input{model.pdf_tex}
  \begin{minipage}[t]{.8\linewidth}
    \centering
    \caption{The basic steganographic model. Alice attempts to send
    message $m$ to Bob, concealing it inside the stegotext $C'$.}
  \end{minipage}
\end{figure}

\section{Security Definitions}
\subsection{Primitives}
Let $P_X$ be a probability mass function with support $\chi$, where $X$
is a discrete random variable taking the values in $\chi$. The
\emph{entropy} of $X$ is
\begin{equation*}
  H(X) = E\l( -\lg P_X \r),
\end{equation*}
where $E(\cdot)$ is the expected value (weighted average) function;
that is,
\begin{equation}
  H(X) = - \sum_{x\in \chi} P_X(x) \lg P_X(x).
  \label{eq:entropy}
\end{equation}
Intuitively, the entropy of $X$ is a measure of the number of bits of
uncertainty in $X$. For example, suppose $\chi$ is the set of all
$n$-bit strings, and $P_X(x) = 1 / 2^n$ for any $x \in \chi$; that
is, every $n$-bit string is equally likely to be pulled from $P_X$.
This would represent a distribution of maximum uncertainty, and it is
straightforward to show that Eqn.~\eqref{eq:entropy} evaluates to $n$
in this case. In fact, $H(X) = \lg \abs{\chi}$ is an upper bound for
$H$, where $\abs \chi$ denotes the cardinality of $\chi$.

The \emph{minimum entropy} of a distribution $P_X$ is defined as
\begin{equation}
  H_\infty\l( X \r) = \min_{x \in \chi} \l\{ -\lg P_X(x) \r\}
  \label{eq:min-entropy}
\end{equation}
This can be understood as a measure of uncertainty for the ``most
probable'' element in $\chi$ according to $P_X$. For example, if there
is some element $x_0$ with $P_X(x_0) = 1$, then $H_\infty(X) = 0$
(there is no uncertainty in $X$). Suppose the most probable element
$x_0$ has probability $P_X(x_0) = 1/2$. Intuitively, the uncertainty
is unity; that is, we can guess that the next value of $X$ will be
$x_0$ to within a single coin flip. Indeed, evaluating
Eqn.~\eqref{eq:min-entropy} for such a distribution shows that
$H_\infty\l( X \r) = 1$.

Given two probability mass functions $P_X$ and $P_Y$, both with
support $\chi$, the \emph{relative entropy}, also called the
Kullback-Leibler divergence, from $P_X$ to $P_Y$ is defined to be
\begin{equation}
  D \l( P_X \concat P_Y \r) =
  \sum_{x \in \chi} P_X(x) \lg \frac{P_X(x)}{P_Y(x)}.
\end{equation}
Intuitively, the relative entropy is a measure of the difference
between $P_X$ and $P_Y$, although it is important to note that it is
not symmetric; that is, $D \l( P_X \concat P_Y \r) \ne D \l( P_Y
\concat P_X \r)$. However, $D \l( P_X \concat P_Y \r) = 0 $ if and
only if $P_X = P_Y$, and increases indefinitely as $P_Y$ diverges from
$P_X$.

A \emph{channel} as defined by \todo{cite} is a distribution on
timestamped bit sequences; i.e., a channel \channel\ is a distribution
with support $\set{\l( \bit, t_1 \r), \l( \bit, t_2 \r), \ldots }$,
where each $t_i \le t_{i+1}$. The intent is to model communication,
where not just the content but also the timing of the communication
may be relevant. Since a particular draw from the distribution
\channel\ depends the history of previously drawn bits, define
$\channel_h$ to be the distribution conditioned on history $h$.
Furthermore, it is useful to think of drawing from the channel in
chunks of $b$ bits at a time, so define $\channel_h^b$ to be the
distribution on the next $b$ bits after the history $h$.

\subsection{Security Definitions}
Hopper, et. al \todo{cite} define a stegosystem $\stg$ as a pair of
randomized algorithms $\l( \stgenc, \stgdec \r)$. $\stgenc$ takes as
input a shared key $k$, a hiddentext message $m$, and a message
history $h$, and an oracle $M(h)$ that samples according to channel
distribution $\channel_h^b$, where channels are required to satisfy,
for all $h$ drawn from \channel, $H_\infty\l( \channel_h^b \r) > 1$.
As output, $\stgenc_k^M\l( m, h \r)$ returns a sequence $c_1 \concat
c_2 \concat \ldots \concat c_\ell$ in the support of $\channel_h^{\ell
b}$. The decryption algorithm $\stgdec_k^M\l( c_1 \concat c_2 \concat
\ldots \concat c_\ell, h \r)$ returns a message $m$, which should be
``correct''; that is, the same message encoded by $\stgenc$, at least
$2/3$ of the time.

The authors define the security of a stegosystem in terms of a game.
The adversarial warden $W$ is given access to $M\l( h \r)$, which
returns draws from $\channel_h^b$, and an oracle $\Ora$. The oracle
$\Ora$ is either $\stgenc_k$ or a function $O(\cdot, \cdot)$, where
$O(m, h)$ simply returns a draw from $\channel_h^{\abs{\stgenc_k\l( m,
h \r)}}$. The warden also has access to randomness $r$. The warden's
advantage against the steganographic secrecy under chosen hiddentext
attack for channel $\channel$ of stegosystem $\stg$ is defined by
Hopper et.\ al to be
\begin{align*}
  \advan_{\stg, \channel}^\sschac \l( W \r) = \abs {
  \Pr_{k, r, M, \stgenc} \l[ W_r ^{M, \stgenc_k(\cdot, \cdot)}
  \text{accepts} \r]
  -
  \Pr_{r, M, O} \l[ W_r ^{M, O(\cdot, \cdot)}
  \text{accepts} \r]
}.
\end{align*}
A stegosystem $\stg$ is \emph{$(t, q, \ell,
\epsilon)$-steganographically secret under chosen hiddenttext attack}
for channel $\channel$ (SS-CHA-$\channel$) if, for any warden $W$
making at most $q$ queries totaling at most $\ell$ bits of hiddentext,
and running in time at most $t$,
\begin{equation*}
  \advan_{\stg, \channel}^\sschac\l( W \r) \le \epsilon;
\end{equation*}
that is, the stegosystem $\stg$ is insecure if an efficient warden can
(with high probability) distinguish between the output of
$\stgenc_k(m, h)$ and draws from $\channel_h^{\abs{\stgenc_k(m,h)}}$,
even when given access to $\channel_h^b$ through $M$. A stegosystem
$\stg$ is \emph{$(t, q, \ell, \epsilon)$-universally
steganographically secret under chosen hiddenttext attack} for channel
$\channel$ (USS-CHA-$\channel$) if it is $\l( t, q, \ell,
\epsilon\r)$-SS-CHA-\channel\ for any channel \channel\ that
satisfies, $\forall h$ drawn from \channel, $H_\infty\l( \channel_h^b
\r) > 1$.

Cachin \todo{cite} takes an information theoretic approach to
steganographic security. He considers the basic prisoner's problem,
where Alice sends either an innocent covertext or a stegotext
concealing a message to Bob over an open communication line. The
warden eavesdrops on the line, and must decide whether the
communication is a covertext or stegotext. Let $P_C$ be the
distribution of covertexts, and let $P_S$ be the distribution of
stegotexts; these distributions are known to the warden. Cachin
defines the overall security of the system in terms of the relative
entropy between $P_C$ and $P_S$; namely, the stegosystem is
\emph{information-theoretic perfectly secure} if
\begin{equation*}
  D\l( P_C \concat P_S \r) = 0,
\end{equation*}
and is \emph{information-theoretic $\epsilon$-secure} if
\begin{equation*}
  D\l( P_C \concat P_S \r) \le \epsilon.
\end{equation*}
Intuitively, Cachin is claiming that a stegosystem is secure if the
probability distributions of covertexts and stegotexts are ``close,"
so that, given a message in the support of $P_C$ and $P_S$, the
warden has very little reason to believe it was drawn from one
distribution over the other. Cachin goes on to analyze the decision
from the framework of hypothesis testing.

Cachin's warden is a much weaker adversary than Hopper's, and
correspondingly, Hopper's SS-CHA-\channel\ game provides a much
stronger definition of security. Some key weaknesses in Cachin's
definition:
\begin{itemize}
  \item Cachin's warden receives a single message over the
    communication line, and must determine her decision. In contrast,
    Hopper's warden has access to an oracle that can be queried as
    much as is computationally feasible. Indeed, Cachin's model is
    roughly equivalent to the SS-CHA-\channel\ game when only a single
    query is allowed.
  \item Cachin's warden knows the distribution of possible messages
    chosen by Alice, but does not know the message. Hopper, however,
    allows for an interactive, chosen hiddentext attack.
  \item Hopper's channels are all conditioned on the \emph{history} of
    previously drawn samples. And since his warden is allowed to
    specify arbitrary history to the oracles, a stegosystem that meets
    SS-CHA-\channel\ security should be secure for any valid history
    of communication on \channel. On the other hand, Cachin's model,
    because it depends on static probability distributions, does not
    capture this sequential element of real communication. Even if
    $P_C$ and $P_S$ were themselves dependent on history, Cachin's
    model would only be able to claim security of the stegosystem for
    the particular history leading up to his experiment.
\end{itemize}

Though Cachin's definitions are weaker, they are not without merit.
Hopper's security schemes require perfect oracles---oracles that can
sample from $\channel_h$ for arbitrary $h$, can be ``rewound," etc.
Given the application, the lack of such an available oracle may make
implementation of security schemes (and proof of their security)
difficult or impossible. In contrast, knowledge about $P_S$ and
$P_C$ might be easier to determine for a scheme couched in Cachin's
framework.

\subsection{Good Stego}

\section{Bad Stego}
\section{Conclusion}
\section{References}

\end{document}
